\chapter{Basis of Quantum Computing}\label{chap:basis of quantum computing}\chaptertoc{}

{\color{palette-main} To do: intro; proof to Theorem 1; proofs of lemmas 1, 2; comments on $n$-qubits gates.}

\section{Quantum computational model}

Inside a quantum system, to perform a computation means to apply some precise operation on a given input state, and to read the output. Such a state is represented by a set of qubits, each one a $2$-level system. Namely, a \textbf{qubit pure state} is a coherent superposition of two states, $\ket{0}$ and $\ket{1}$,
\[
    \ket{\psi} = a \ket{0} + b \ket{1}
    \qq{normalized:}
    \abs{a}^2 + \abs{b}^2 = 1
\]
with complex coefficients $a,b \in\C$. Note that if the qubits are many, the state of one particular qubit may not be pure, but entangled.

\begin{figure}
    \centering
    \def\rotationSphere{-120}
\def\tiltsphere{20}
\def\radiusSphere{2cm}
\def\psiLat{50}
\def\psiLon{60}
\begin{blochsphere}[
    radius=\radiusSphere,
    tilt=\tiltsphere,
    rotation=\rotationSphere,
    opacity=0
    ]
    % --- Ball setup ---
    \drawBallGrid[style={opacity=0.05}]{30}{45}
    \drawLongitudeCircle[style={dashed,color=palette-gray}]{-\psiLon}
    \drawLatitudeCircle[style={dashed,color=palette-gray}]{0}
    
    % --- Points ---
    \labelLatLon{ket0}{90}{0};
    \labelLatLon{ket1}{-90}{0};
    \labelLatLon{ketminus}{0}{180};
    \labelLatLon{ketplus}{0}{0};
    \labelLatLon{ketpluspi2}{0}{-90};
    \labelLatLon{ketplus3pi2}{0}{-270};
    \labelLatLon{psi}{\psiLat}{-\psiLon};
    
    % --- Axis ---
    \draw[-stealth]
    (0,0) -- (ket0) node[anchor=south,inner sep=0.5em] 
        at (ket0) {\footnotesize $z$};
    \draw[-stealth] 
    (0,0) -- (ketplus) node[anchor=north east,inner sep=0.25em] 
        at (ketplus) {\footnotesize$x$};
    \draw[-stealth] 
    (0,0) -- (ketpluspi2) node[anchor=west,inner sep=0.5em] 
        at (ketpluspi2) {\footnotesize $y$};
        
    % --- Ket ---
    \draw[color=palette-main,-stealth] 
        (0,0) -- (psi) node[right,xshift=0.3em]{\footnotesize $\ket{\psi}$};
    
    % --- Angles ---
    \coordinate (origin) at (0,0);
    {
        \setDrawingPlane{0}{0}
        \draw[current plane,dashed,color=palette-main] 
            (0,0) -- (-90+\psiLon:{cos(\psiLat)*\radiusSphere}) coordinate (psiProjectedEquat) -- (psi);
        \pic[current plane,fill=palette-main,fill opacity=0.5,text opacity=1,"\footnotesize $\varphi$",angle eccentricity=2.2]
            {angle=ketplus--origin--psiProjectedEquat};
    }
    {
        \setLongitudinalDrawingPlane{-\psiLon}
        \pic[current plane,fill=palette-main,fill opacity=0.5,text opacity=1,"\footnotesize $\theta$",angle eccentricity=1.5]
            {angle=psi--origin--ket0};
    }
\end{blochsphere}
    \caption{Representation of the Bloch's sphere; apart from an unmeasurable global phase, a point on its surface is a well defined $1$-qubit state.}
    \label{fig:bloch sphere}
\end{figure}

Notice that a global phase over the pure state $\ket{\psi}$ is not measurable; thus we can choose $a$ to be real, only accounting for the relative phase $e^{i \varphi}$,
\[
    \ket{\psi} = a \ket{0} + e^{i\varphi} \sqrt{1-a^2} \ket{1}
\]
Being $a^2 < 1$, we introduce the angle $\theta$ such that
\[
    \ket{\psi} = \cos \lrR{\frac{\theta}{2}} \ket{0} + e^{i\varphi} \sin \lrR{\frac{\theta}{2}} \ket{1}
\]
$\varphi$ and $\theta$ have the following meaning: a single qubit pure state can be represented as a point on the surface of a \textbf{Bloch's sphere} with azimuthal angle $\theta$ and polar angle $\varphi$. A representation of such parametrization is given in Fig.~\ref{fig:bloch sphere}.

Under this parametrization, evidently, the state $\ket{0}$ is obtained for $\theta=0$ and generic $\varphi$; analogously, for $\theta=\pi$ and generic $\varphi$ the state $\ket{1}$ is realized. Then $\ket{0}$ occupies the ``north pole'' of the Bloch's sphere, and $\ket{1}$ the ``south pole'', as depicted in Fig.~\ref{fig:bloch sphere poles}. We also define the states $\ket{+}$ and $\ket{-}$,
\[
    \ket{+} = \frac{\ket{0} + \ket{1}}{\sqrt{2}}
    \qquad
    \ket{-} = \frac{\ket{0} - \ket{1}}{\sqrt{2}}
\]
antipodal as well. Next section is devoted to a brief discussion of those operations mapping a pure state to a pure state acting only on the qubit Hilbert space.

\begin{figure}
    \centering
    \def\rotationSphere{-120}
\def\tiltsphere{20}
\def\radiusSphere{2cm}
\begin{blochsphere}[
    radius=\radiusSphere,
    tilt=\tiltsphere,
    rotation=\rotationSphere,
    opacity=0
    ]
    % --- Ball setup ---
    \drawBallGrid[style={opacity=0.05}]{30}{45}
    
    % --- Points ---
    \labelLatLon{ket0}{90}{0};
    \labelLatLon{ket1}{-90}{0};
    \labelLatLon{ketminus}{0}{180};
    \labelLatLon{ketplus}{0}{0};
    \labelLatLon{ketpluspi2}{0}{-90};
    \labelLatLon{ketplus3pi2}{0}{-270};

    % --- Antipodal axes ---
    \draw[thin,color=palette-gray,dashed]
        (ket0) -- (ket1)
        (ketplus) -- (ketminus)
        (ketpluspi2) -- (ketplus3pi2);

    % --- States ---
    \filldraw
        (ket0) circle (1pt) node[anchor=south,inner sep=0.5em] {$\ket{0}$}
        (ket1) circle (1pt) node[anchor=north,inner sep=0.5em] {$\ket{1}$}
        (ketplus) circle (1pt) node[anchor=north east,inner sep=0.25em] {$\ket{+}$}
        (ketminus) circle (1pt) node[anchor=south west,inner sep=0.25em] {$\ket{-}$}
        (ketpluspi2) circle (1pt)
        (ketplus3pi2) circle (1pt);
\end{blochsphere}
    \caption{The ``north pole'' of the sphere represents, apart from a general $U(1)$ symmetry, the state $\ket{0}$; the ``south pole'' analogously represents the state $\ket{1}$. Hadamard states $\ket{\pm}$ are well-defined (both $\theta$ and $\varphi$ have a precise value).}
    \label{fig:bloch sphere poles}
\end{figure}

\begin{tcolorbox}[ colframe=palette-main, title= \textbf{Computation is a physical process...}] %comment col/ till we set palette refs
    Without physical means, computation would not be possible: information is always stored, transmitted, and processed by \textit{something}. This may seem obvious, but in classical computation theory, researchers did not generally consider physical aspects. All the pioneers you can think of pursued a computational theory through intuition and logic alone, which led them to mistaken assumptions about abstract and ``self-evident'' foundations.
    Thus, we must abandon a purely logical notion of computation and, starting now (\textit{i.e.,} from this very line, as a mantra: ``\textit{Computation is a physical process. Computation is a physical process.}''), we should think of it as a process governed by the laws of physics. As a mere consequence, we find that quantum computation has given us a new perspective on the laws of nature, and thus a new approach to computational research.
\end{tcolorbox}

\subsection{$1$-qubit gates}

We introduce \textbf{$1$-qubit operations}: these are simply those mapping a pure state $\ket{\psi}$ into another pure state - namely, another point od the sphere surface. Notice that, given the single-qubit Hilbert space
\[
    \Hilbert_1 =  \Span \lbrace \ket{0}, \ket{1} \rbrace
\]
a \textbf{$1$-qubit operator} $U_1$ simply acts by
\[
    U_1 \colon \Hilbert_1 \to \Hilbert_1
\]
meaning that it is precisely representable as a \textbf{rotation} of the vector representing the state in the Bloch's sphere. Let $\sigma^i$, for $i=x,y,z$, be the Pauli matrix,
\[
    \sigma^x \equiv \begin{bmatrix} & 1 \\ 1 & \end{bmatrix}
    \qquad
    \sigma^y \equiv \begin{bmatrix} & -i \\ i & \end{bmatrix}
    \qquad
    \sigma^z \equiv \begin{bmatrix} 1 & \\ & -1 \end{bmatrix}
\]
The common, lighter notation is
\[
    X = \sigma^x
    \qquad
    Y = \sigma^y
    \qquad
    Z = \sigma^z
\]
Let $\Rotation_\nu (\theta)$ be a rotation of angle $\theta$ around axis $\nu$, intended in the Bloch's sphere framework. Then
\[
\begin{aligned}
    \Rotation_x (\theta) =& e^{-i \theta \sigma^x /2 } = \cos \lrR{\frac{\theta}{2}} \Id - i \sin \lrR{\frac{\theta}{2}} X = \begin{bmatrix}
        \cos(\theta/2) & -i \sin(\theta/2) \\
        i \sin(\theta/2) & \cos(\theta/2)
    \end{bmatrix} \\
    \Rotation_y (\theta) =& e^{-i \theta \sigma^y /2 } = \cos \lrR{\frac{\theta}{2}} \Id - i \sin \lrR{\frac{\theta}{2}} Y = \begin{bmatrix}
        \cos(\theta/2) & \sin(\theta/2) \\
        - \sin(\theta/2) & \cos(\theta/2)
    \end{bmatrix} \\
    \Rotation_z (\theta) =& e^{-i \theta \sigma^z /2 } = \cos \lrR{\frac{\theta}{2}} \Id - i \sin \lrR{\frac{\theta}{2}} Z = \begin{bmatrix}
        e^{i\theta/2} & \\
        & e^{-i\theta/2}
    \end{bmatrix} \\
\end{aligned}
\]
so, composing these matrices \textit{somehow}, we should be able to represent any rotation, thus any $1$-qubit operator. This is true indeed. Consider in fact the following
\begin{theorem}
    Let $U_1$ be a $1$-qubit operator. Then four angles $\alpha$, $\beta$, $\gamma$ and $\delta$ exists such that
    \[
        U_1 = e^{i\alpha} \Rotation_z(\beta) \Rotation_y (\gamma) \Rotation_z(\delta)
    \]
    Moreover, defining the operators $A$, $B$ and $C$ as
    \[
        A \equiv \Rotation_z (\beta) \Rotation_y \lrR{\frac{\gamma}{2}}
        \quad
        B \equiv \Rotation_y \lrR{-\frac{\gamma}{2}} \Rotation_z \lrR{- \frac{\delta + \beta}{2}}
        \quad
        C \equiv \Rotation_z \lrR{\frac{\delta - \beta}{2}}
    \]
    the following identities hold:
    \[
        ABC = \Id
        \qquad
        e^{i\alpha} AXBXC = U_1
    \]
\end{theorem}

\begin{proof}
    \Comment{Add: proof.}
\end{proof}

The standard representation of quantum operations on qubits is the \textbf{quantum circuit representation}. Here, the Hilbert space is represented by a line over which a state ``lives''. To apply a $1$-qubit gate means to insert at some point along the line a box representing the operator. The line is to be read sequentially from left to right: the operation
\[
    U_1 \colon \ket{\psi} \to\ \ket{\phi} \equiv U_1 \ket{\psi}
\]
is represented by
\begin{center}
    \begin{quantikz}
        \lstick{$\ket{\psi}$} & \gate{U_1} & \rstick{$\ket{\phi}$}
    \end{quantikz}
\end{center}
We now list some relevant $1$-qubit gates.

\paragraph{Hadamard gate}

The \textbf{Hadamard gate} is defined as
\[
    H \equiv \frac{1}{\sqrt{2}} \begin{bmatrix}
        1 & 1  \\
        1 & -1
    \end{bmatrix}
\]
It acts simply on the computational basis,
\[
    H \ket{0} = \frac{\ket{0} + \ket{1}}{\sqrt{2}} = \ket{+}
    \qquad
    H \ket{1} = \frac{\ket{0} - \ket{1}}{\sqrt{2}} = \ket{-}
\]
and similarly
\[
    H \ket{+} = \ket{0}
    \qquad
    H \ket{-} = \ket{1}
\]
It follows immediately, and can be easily checked, $H^2 = \Id$.

\paragraph{$T$-gate} Also referred to as the $\pi/8$-gate, the \textbf{$T$-gate} is defined as
\[
    T \equiv \begin{bmatrix}
        1 & \\
          & e^{i\pi/4}
    \end{bmatrix} = e^{i\pi/8} \begin{bmatrix}
        e^{-i\pi/8} & \\
                    & e^{i\pi/8}
    \end{bmatrix}
\]
Notice that the global $e^{i\pi/8}$ phase in front of the gate, for a single-qubit process, is global and immeasurable. The gate dephases the computational basis as
\[
    T \ket{0} = \ket{0}
    \qquad
    T \ket{1} = e^{i\pi/4} \ket{1}
\]

Interesting results are the following
\begin{lemma}
    The set of operators
    \[
        \left\lbrace
            HTHT, THTH
        \right\rbrace
    \]
    can compose \textbf{any arbitrary Bloch rotation}, thus represent a \textbf{universal set of $1$-qubits gates}.
\end{lemma}
\begin{lemma}
    The set of $1$-qubit operators
    \[
        \left\lbrace 
            H, T
        \right\rbrace
    \]
    is \textbf{universal}.
\end{lemma}

\subsection{$2$-qubits gates}\label{subsec:2-qubits gates}

Consider two isolated qubits.  Let the global Hilbert space be $\Hilbert_2 \equiv \Hilbert_1 \otimes \Hilbert_1$. We call the ``first qubit'' the one relative to the first space in the outer product, and the ``second qubit'' the other one. A $2$-qubit operation $U_2$ acts by
\[
    U_2 \colon \Hilbert_2 \to \Hilbert_2
\]
Notice that only a small portion of the couple space $\Hilbert_2$ is made of \textbf{separable states}, which are, states that can be written as products of single-qubit pure states. The vast majority of states in $\Hilbert_2$ are \textbf{entangled}. Graphically,
\begin{center}
    \begin{quantikz}
        \lstick[2]{$\ket{\psi}$} & \gate[2]{U_2} & \rstick[2]{$\ket{\phi}$} \\
        &&
    \end{quantikz}
\end{center}
Namely, a $2$-qubit gate performs some operation on the collective state. In quantum computing, a very important class of $2$-qubits gates are those \textbf{controlled}. For such gates, one of the input states is the \textbf{control} and the other is the \textbf{target}. The control is in general a quantum mixture of single-qubit states $\ket{0}$ and $\ket{1}$. Now, the general form of the $2$-qubits state is clearly
\[
    \ket{\psi} = a_{00} \ket{00} + a_{01} \ket{01} + a_{10} \ket{10} + a_{11} \ket{11}
\]
with $a_{ij} \in \C$ (normalized) and $\ket{ij} = \ket{i} \otimes \ket{j}$. Let the first qubit be the control and the second be the target. Then we can rewrite
\[
    \ket{\psi} = \ket{0} \otimes \lrS{\vphantom{A^A} a_{00} \ket{0} + a_{01} \ket{1}} + \ket{1} \otimes \lrS{\vphantom{A^A} a_{10} \ket{0} + a_{11} \ket{1}}
\]
A controlled gate acts on the target qubit as an identity if the control is zero, and as some $1$-qubit operator $U_1$ if the control is one,
\[
    U_2 \ket{\psi} = \ket{0} \otimes \lrS{\vphantom{A^A} a_{00} \ket{0} + a_{01} \ket{1}} + \ket{1} \otimes U_1 \lrS{\vphantom{A^A} a_{10} \ket{0} + a_{11} \ket{1}}
\]
clearly equivalent to
\[
    U_2 = \op{0} \otimes \Id + \op{1} \otimes U_1
\]
The graphic representation of the controlled-$U_1$ gate is
\begin{center}
    \begin{quantikz}
        & \ctrl{1} & \\
        & \gate{U_1} &
    \end{quantikz}
\end{center}

The matrix representation of a $2$-qubits operator is a $2^2 \times 2^2$ matrix,
\[
    \lrS{U_2}_{\alpha\beta} = \mel{\alpha}{U_2}{\beta}
\]
with $\ket{\alpha}$, $\ket{\beta}$ the $2$-qubits orthonormal states. The computational basis through which we may express $U_2$ is
\[
    \left\lbrace \ket{00},\ket{01},\ket{10},\ket{11} \right\rbrace
\]
Namely: each entry of the $4\times 4$ matrix can be identified by four logical numbers,
\[
    \lrS{U_2}_{(ij)(lm)} = \mel{ij}{U_2}{lm}
\]
Then its controlled version is
\[
\begin{aligned}
    \lrS{U_2}_{(ij)(lm)} &= \mel{ij}{ \lrS{\vphantom{A^A}\op{0} \otimes \Id + \op{1} \otimes U_1} }{lm} \\
    &= \delta_{i0}\delta_{l0} \delta_{jm} + \delta_{i1}\delta_{l1} \lrS{U_1}_{jm} 
\end{aligned}
\]
which gives
\[
    \lrS{U_2} = \begin{bmatrix}
        1 & & & \\
        & 1 & & \\
        & & U_{00} & U_{01} \\
        & & U_{10} & U_{11}
    \end{bmatrix}
\]

A very simple and useful example of controlled gate is the \textbf{controlled not gate}, also referred to as $\cnot$.

\begin{eg}[Controlled not]
    In classical computing, the NOT operator simply logically inverts the bit state. Its quantum analog is evidently the $X$ $1$-qubit operator, inverting the amplitudes for the $0$ and the $1$ state. Then the $\cnot$ gate is simply
    \[
        \cnot = \op{0} \otimes \Id + \op{1} \otimes X
    \]
    graphically represented by
    \begin{center}
        \begin{quantikz}
            \lstick{$c$} & \ctrl{1} & \\
            \lstick{$t$} & \gate{X} &
        \end{quantikz}
    \end{center}
    with $c$ the control line and $t$ the target line. More often, the gate is represented as:
    \begin{center}
        \begin{quantikz}
            & \ctrl{1} & \\
            & \targ{} &
        \end{quantikz}
    \end{center}
    Its matrix representation is immediately computable,
    \[
        \cnot = \begin{bmatrix}
        1 & & & \\
        & 1 & & \\
        & & & 1 \\
        & & 1 & 
    \end{bmatrix}
    \]
    We could have guessed: the $X$ inversion is active only on the portion of the pair Hilbert space $\Hilbert_2$ for which the first qubit is ``logically active''.
\end{eg}

\subsection{$n$-qubits gates}

It is pretty much straightforward to extend the previous definitions to $n$ qubits. \Comment{Go on...}

\section{A protocol for quantum algorithms}

Now, One could correctly think: what kind of advantage do we gain by using Quantum Mechanics? This latter is essentially a very special (non-abelian) theory of probability. This brief section highlights the main difference between probabilistic classical computing and quantum computing.

\subsection{Classical probabilistic model}

\begin{figure}
    \centering
    \def\hspacing{4}
\def\vspacing{2}
\begin{tikzpicture}
    \node (a0) at (0,0) {};
    \foreach \i in {0,1,...,4}{
        \node (b\i) at (\hspacing,{(\i-2)*\vspacing}) {};
        \node (c\i) at ({2*\hspacing},{(\i-2)*\vspacing}) {};
    }

    \draw
        \foreach \i in {0,1,...,4}{
            (a0) -- (b\i) 
                node [near end,anchor=north,yshift=-0.25em,xshift=0.1em] {$p_{0\i}^{(1)}$}
        }
        (b0) -- (c0)
            node [near start,anchor=north,yshift=-0.25em] {$p_{00}^{(2)}$}
        (b1) -- (c0)
            node [near start,anchor=north,yshift=-0.25em] {$p_{01}^{(2)}$}
        (b4) -- (c4)
            node [near start,anchor=north,yshift=-0.25em] {$p_{44}^{(2)}$}
        (b3) -- (c4)
            node [near start,anchor=north,yshift=-0.25em] {$p_{34}^{(2)}$}
        ;
    
    \filldraw[color=palette-main,fill=white] 
        (a0) circle (1em)
            node () {$0$};
    \filldraw[color=palette-main,fill=white]
        \foreach \i in {0,1,2,3,4}{
            (b\i) circle (1em)
                node {$\i$}
            (c\i) circle (1em)
                node {$\i$}
        };

    \node at ({1.5*\hspacing},0) {$\vdots$};
    \node at ({2.5*\hspacing},0) {$\cdots$};
            
\end{tikzpicture}
    \caption{A probabilistic ``network-like'' model of computations. The circled numbers represent the logical states of the system; the meaning of other symbols is explained in the text. Here we suppose the system to have $4$ logical states, and to start from the logical state ``$0$''.}
    \label{fig:probabilistic model}
\end{figure}

In general, we may imagine a classical probabilistic protocol of computation as a sort of network: given some input and some set of states, each computational step can be represented as a mapping of the input state with fixed probability, as represented in Fig.\ref{fig:probabilistic model}. We enumerate the logical states of the system, $\lbrace 0,1,\cdots \rbrace$, and define
\[
    p_{ij}^{(k)} \equiv \text{probability of transitioning from state $i$ to state $j$ at step $k$.}
\]

The obvious consequence is that the probability of ending on state $j$ starting from state $i$ in $n$ steps of computation is the sum of all possible paths in the diagram,
\[
    \prob{i \to j; n} = \sum_{k_1} p_{ik_1}^{(1)} \sum_{k_2} p_{k_1 k_2}^{(2)} \cdots \sum_{k_{n-1}} p_{k_{n-1} j}^{(n)}
\]
e.g. for the simple situation in Fig.\ref{fig:probabilistic model} we have
\[
    \prob{0 \to 1;2} = \sum_k p_{0k}^{(1)} p_{k1}^{(2)}
\]

The main feature of this scheme can be synthesized as follows: the more the paths connecting the initial and the final state, the bigger (or at least the ``not-lower'') the resulting transition probability. This is because each probability, taken singularly, is positive-defined. As we shall see immediately, this is not the case on quantum hardware.

\subsection{Quantum model}

\begin{figure}
    \centering
    \input{introduction to quantum computation/chapters/basis of quantum computing/pictures/quantum model}
    \caption{Quantum ``network-like'' model of computation. This figure is the direct extension of Fig.~\ref{fig:probabilistic model}, with analogous significance of symbols.}
    \label{fig:quantum model}
\end{figure}

The quantum model is very similar, and represented in Fig.~\ref{fig:quantum model}. Here the logical states are quantum states in some Hilbert space, and the transition probabilities are replaced by transition \textit{amplitudes}. The amplitude $\alpha_{ij}^{(k)}$ represent transition from state $\ket{i}$ to state $\ket{j}$ at step $k$ of computation. Here, evidently
\[
    \prob{i \to j; n} = \left| \sum_{k_1} \alpha_{ik_1}^{(1)} \sum_{k_2} \alpha_{k_1 k_2}^{(2)} \cdots \sum_{k_{n-1}} \alpha_{k_{n-1} j}^{(n)} \right|^2
\]
but amplitudes are complex numbers, thus can destructively interfere! For a quantum protocol, it holds no more that the more two states are connected, the higher the respective transition probability.

Take for example the very simple two-states system in Fig.~\ref{fig:two states quantum model}, and suppose
\[
    \alpha^{(1)}_{00} = \alpha^{(1)}_{01} = t
    \qquad
    \alpha^{(2)}_{00} = t
    \qquad
    \alpha^{(2)}_{10} = -t
\]
This means evidently
\[
    \prob{0 \to 0;2} = \alpha^{(1)}_{00} = \alpha^{(2)}_{00} + \alpha^{(1)}_{01} \alpha^{(2)}_{10} = t^2 - t^2 = 0
\]
The probability of transitioning from state $0$ to the same state in two steps is nullified because of destructive interference. This is the most real relevant difference with respect to classical probabilistic computing: the unitary processes that implement computation act on amplitudes, thus by cleverly designing algorithm we can take advantage of interference itself -- the great deal of Quantum Mechanics -- in order to engineer precise behaviors of the computer.

\begin{figure}
    \centering
    \def\hspacing{4}
\def\vspacing{2}
\begin{tikzpicture}
    \foreach \i in {0,1}{
        \node (a\i) at (0,{(\i-2)*\vspacing}) {};
        \node (b\i) at (\hspacing,{(\i-2)*\vspacing}) {};
        \node (c\i) at ({2*\hspacing},{(\i-2)*\vspacing}) {};
    }

    \draw[dashed]
        (a1) -- (b0) -- (c1)
        (a1) -- (b1) -- (c1);
    \draw[color=palette-main]
        (a0) -- (b0) node[near start,anchor=north] {$t$} -- (c0) node[near end,anchor=north] {$t$}
        (a0) -- (b1) node[near start,anchor=south] {$t$} -- (c0) node[near end,anchor=south] {$-t$};

    \filldraw[color=palette-main,fill=white]
        \foreach \i in {0,1}{
            (a\i) circle (1em)
                node {$\ket{\i}$}
            (b\i) circle (1em)
                node {$\ket{\i}$}
            (c\i) circle (1em)
                node {$\ket{\i}$}
        };            
\end{tikzpicture}
    \caption{A two-step computation on a single qubit computational model. As explained in text, we here suppose the amplitudes to be $\braket{0}{0} = \braket{0}{1} = t$ at the first step and $\braket{0}{0} = t$, $\braket{1}{0} = - t$ at the second step. The dashed lines are connections irrelevant to the process $0\to 0$ in two steps, while the green lines are the relevant links.}
    \label{fig:two states quantum model}
\end{figure}

\Comment{Go on...}